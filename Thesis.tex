\documentclass{mimosis}

\usepackage{sidenotes}

% letter size
\KOMAoptions{paper=letter}

\newacronym[description={Principal component analysis}]{PCA}{PCA}{principal component analysis}
\newacronym                                            {SNF}{SNF}{Smith normal form}
\newacronym[description={Topological data analysis}]   {TDA}{TDA}{topological data analysis}

\newglossaryentry{LaTeX}{%
  name        = {\LaTeX},
  description = {A document preparation system},
  sort        = {LaTeX},
}

\newglossaryentry{Real numbers}{%
  name        = {$\real$},
  description = {The set of real numbers},
  sort        = {Real numbers},
}

\makeindex
\makeglossaries

%%%%%%%%%%%%%%%%%%%%%%%%%%%%%%%%%%%%%%%%%%%%%%%%%%%%%%%%%%%%%%%%%%%%%%%%
% Title Page Information
%%%%%%%%%%%%%%%%%%%%%%%%%%%%%%%%%%%%%%%%%%%%%%%%%%%%%%%%%%%%%%%%%%%%%%%%

\title{\textbf{Computational Approaches for Acoustic \& Environmental Informational Utility in Marine Robotics}}
\author{EeShan Chetan Bhatt}
\date{May 4\th 2021}

\begin{document}

\mainmatter

  \def\signature#1#2{\par\noindent#1\dotfill\null\\*
  {\raggedleft #2\par}}

%% --- TITLEPAGE --- %%

% keep this to protect internal macros + use @ as a normal character
\makeatletter

\begin{titlepage}
  \begin{center}
    \begin{Large}
      \@title
    \end{Large}\\[0.1em]
    %
    \emph{\footnotesize by}\\
    {\large \@author} \\[-0.25em]
    B.S.E. Mechanical Engineering, \textsc{Duke University} (2015) \\ [2em]
    %
    \begin{singlespace}
    {Submitted to the Joint Program in Oceanography and Applied Ocean Science \& Engineering in partial fulfillment of the requirements for the degree of Doctor of Philosophy in Mechanical \& Oceanographic Engineering} \\
    \end{singlespace}
    %
    \emph{\footnotesize at the}\\
    {\large \textsc{Massachusetts Institute of Technology}} \\
    \emph{\footnotesize and the}\\
    {\large \textsc{Woods Hole Oceanographic Institution}} \\ [2em]
    %
    \begin{singlespace}
    {\copyright2021 E.C. Bhatt. All rights reserved. \\ The author hereby grants to MIT and WHOI permission to reproduce and to distribute publicly copies of this thesis document in whole or in part in any medium now known or hereafter created.} \\ [2em]

    \signature{Author}{\footnotesize Department of Mechanical Engineering, MIT \\ Applied Ocean Science \& Engineering, WHOI \\ \@date}
    \vspace{1em}
    \signature{Certified by}{Henrik Schmidt \\ \footnotesize Professor of Mechanical and Ocean Engineering, MIT \\ Thesis Supervisor}
    \vspace{1em}
    \signature{Accepted by}{Nicholas Hadjiconstantinou \\ \footnotesize Professor of Mechanical Engineering, MIT \\ Chair, Department Committee on Graduate Students}
    \vspace{1em}
    \signature{Accepted by}{David Ralston \\ \footnotesize Associate Scientist with Tenure, Applied Ocean Physics \& Engineering, WHOI \\ Chair, Joint Committee for Applied Ocean Science \& Engineering}
    \end{singlespace}
  \end{center}
  \makeatother
\end{titlepage}

\newpage
\null
\thispagestyle{empty}
\newpage
  %% ABSTRACT %%
\begin{center}
{\large \@title} \\
\emph{\footnotesize by} \\
\@author \\
\end{center}
%
\noindent \textbf{ABSTRACT} %less than 200 words for WHOI (350 for MIT)

\noindent Enim blandit volutpat maecenas volutpat blandit. Sit amet mattis vulpu
tate enim nulla aliquet porttitor lacus. Purus semper eget duis at tellus at urn
a condimentum mattis. Vitae justo eget magna fermentum iaculis eu. Magnis dis pa
rturient montes nascetur ridiculus mus mauris vitae ultricies. Fringilla phasell
us faucibus scelerisque eleifend donec. Sit amet aliquam id diam maecenas. Ut fa
ucibus pulvinar elementum integer. Suspendisse sed nisi lacus sed viverra tellus
 in hac. Tortor at auctor urna nunc id cursus metus. Semper viverra nam libero j
usto laoreet sit. Dolor sit amet consectetur adipiscing elit. Neque aliquam vest
ibulum morbi blandit cursus. Aliquam sem fringilla ut morbi tincidunt augue inte
rdum velit. Dolor magna eget est lorem ipsum dolor sit amet. Consequat ac felis 
donec et odio pellentesque diam volutpat. Sit amet aliquam id diam maecenas ultr
icies. Viverra mauris in aliquam sem fringilla. Venenatis lectus magna fringilla
 urna porttitor. Risus viverra adipiscing at in tellus integer feugiat scelerisq
ue. \\

\begin{singlespace}
\small{
\noindent Thesis Supervisor: Thesis Supervisor \\
\noindent Title: Thesis Supervisor Title}
\end{singlespace}

  \tableofcontents

  \include{Sources/Introduction}

% This ensures that the subsequent sections are being included as root
% items in the bookmark structure of your PDF reader.
\bookmarksetup{startatroot}
\backmatter

  \begingroup
    \let\clearpage\relax
    \glsaddall
    \printglossary[type=\acronymtype]
    \newpage
    \printglossary
  \endgroup

  \printindex
  \printbibliography

\end{document}
